\documentclass[a4paper]{article}

\usepackage[english]{babel}
\usepackage[utf8]{inputenc}
\usepackage{amsmath}
%\usepackage{graphicx}

\title{\textbf{Artificial Intelligence \\
    Uppsala University -- Autumn 2014 \\
    Report for Assignment~$2$
    by Team~$1$
  }
}

\author{Sander Cox \and Martin Kjellin \and Malin Lundberg \and Sverrir
  Thorgeirsson}

\date{\today}

\begin{document}
\maketitle

\section{Introduction}

This report describes the algorithms that are used in our program to compete
on the Where's Croc game. The goal of the game is to find the crocodile Croc,
which can be located at any of 35 waterholes in Wollomunga national
park. Hopefully, he will be found before he eats the two backpackers that roam
the park. The search is performed using a \emph{hidden Markov model} (HMM).

\section{Estimating Croc's Location}

In order to find Croc, we first estimate his most probable current
location using the so-called \emph{state estimation algorithm}. Then, we use
the result of this estimation to decide where to go next (see section~\ref{hunt}).

In a HMM, the state of the world at time $t$ is
described by a state variable $S_t$. We cannot observe this state
directly. Instead, we have access to a number of observation (evidence)
variables $O^x_t$ which indirectly provide us with information about the state
of the world.

There are two types of probabilities in a HMM: transition probabilities and
observation probabilities. %Check these terms
Given that the world was in state $S_{t-1}$ at time $t-1$, it
will be in state $S_t$ at time $t$ with transition probability
$P(S_t|S_{t-1})$. Given that the world is in state $S_t$, we will observe
$O^x_t$ with observation probability $P(O^x_t|S_t)$.

In this case, the state of the world is Croc's current location (waterhole
number), so $S_t \in \{1, 2, \dots, 35\}$. %check indexing
The observation variables $O^c_t$, $O^s_t$, and $O^a_t$ are the current
readings (of calcium, salinity, and alcalinity, respectively) from the sensor
that Croc is carrying.

We want to calculate $P(S_t|S_{t-1}, O^c_t, O^s_t, O^a_t)$ for each state,
that is, the probability of Croc currently lurking at that waterhole
given his previous location and the current sensor readings. This probability
is proportional to
\begin{equation*}
P(O^c_t|S_t)P(O^s_t|S_t)P(O^a_t|S_t)P^*(S_t|S_{t-1})
\end{equation*}
where
\begin{equation*}
P^*(S_t|S_{t-1}) = \sum\limits_{i=1}^{35} P(S_t|S_{t-1}=i)P(S_{t-1}=i)  
\end{equation*}
(that is, a sum over all possible previous states).

The recursive character of
the transition probabilities calls for a base case, that is, a value $S_0$ for
each state. If the two backpackers, when the game starts, are at two different
waterholes and are not being eaten, we set $P(S_0) = 0$ for the waterholes
where they are and $P(S_0) = \frac{1}{33}$ for all others. That is, we assume
that the probability is equally distributed over all waterholes except the
ones where Croc is apparently not currently located.

The observation probabilities must be calculated from the sensor
readings. Since we know, for each waterhole, the expectations $\mu$ and the
standard deviations $\sigma$ of the distributions from which the readings were
drawn, we can calculate the value of the probability density function
$f(O^x_t)$ for each observation $O^x_t$ using the formula
\begin{equation*}
  f(O^x_t) = \frac{1}{\sigma \sqrt{2 \pi}}e^{-\frac{(O^x_t-\mu)^2}{2\sigma ^2}}
\end{equation*}
In a slight abuse of probability theory, these values can then be used as
observation probabilities. That is, we assume that
\begin{equation*}
  P(O^x_t|S_t) = f(O^x_t)
\end{equation*}

The values calculated as
\begin{equation*}
  P(O^c_t|S_t)P(O^s_t|S_t)P(O^a_t|S_t)P^*(S_t|S_{t-1})
\end{equation*}
are not probabilities, that is, their sum is not 1. Therefore, they are
normalized by being divided by the sum of the calculated values for all
states. The results of the normalization are the desired probabilities $P(S_t|S_{t-1}, O^c_t, O^s_t, O^a_t)$.

\section{Hunting Croc}
\label{hunt}

This section will describe what we do with the information about Croc's
probable current location.

\end{document}
