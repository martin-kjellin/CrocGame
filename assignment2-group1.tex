\documentclass[a4paper]{article}

\usepackage[english]{babel}
\usepackage[utf8]{inputenc}
\usepackage{amsmath}
%\usepackage{graphicx}

\title{\textbf{Artificial Intelligence \\
    Uppsala University -- Autumn 2014 \\
    Report for Assignment~$2$
    by Team~$1$
  }
}

\author{Sander Cox \and Martin Kjellin \and Malin Lundberg \and Sverrir
  Thorgeirsson}

\date{\today}

\begin{document}
\maketitle

\section{Introduction}

Lorem ipsum dolor sit amet ...

\section{Hidden Markov Models and the State Estimation Algorithm}

In order to find Croc, we will first estimate his most probable current
location using the state estimation algorithm. Then, we will use the result of
this estimation to decide where (in what direction) to go next.

In a hidden Markov model (HMM), the state of the world at time $t$ is described by a
state variable $S_t$. We cannot observe this state directly. Instead, we have
access to a number of observation (evidence) variables $O^i_t$ which
indirectly provide us with information about the state of the world.

There are two types of probabilities in a HMM: transition probabilities and
observation probabilities. %Check these terms
Given that the world was in state $S_{t-1}$ at time $t-1$, it
will be in state $S_t$ at time $t$ with transition probability
$P(S_t|S_{t-1})$. Given that the world is in state $S_t$, we will observe
$O^i_t$ with observation probability $P(O^i_t|S_t)$. %Check the formulae

In this case, the state of the world is Croc's current location (waterhole number), so $S_t \in
\{1, 2, \dots, 35\}$. %check indexing
The observation variables $O^c_t$, $O^s_t$, and $O^a_t$ are the current
readings (of calcium, salinity, and alcalinity) % check these
from the sensor that Croc is
carrying. These readings do not give us the observation probabilities
directly, but since we know, for each waterhole, the expectations $\mu$ and the
standard deviations $\sigma$ of the distributions from which the readings were
drawn, we can calculate the value of the probability density function
$f(O^i_t)$ for each observation $O^i_t$ using the formula
\begin{equation*}
  f(O^i_t) = \frac{1}{\sigma \sqrt{2 \pi}}e^{-\frac{(x-\mu)^2}{2\sigma ^2}}
\end{equation*}
These values can then be used as observation probabilities in the calculation
of the probabilities $P(S_t|S_{t-1}, O^i_t)$ for each state. % To be
                                % continued, check ...

\section{Yet Another Section}

\end{document}
