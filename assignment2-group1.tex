\documentclass[a4paper]{article}

\usepackage[english]{babel}
\usepackage[utf8]{inputenc}
\usepackage{amsmath}
%\usepackage{graphicx}

\title{\textbf{Artificial Intelligence \\
    Uppsala University -- Autumn 2014 \\
    Report for Assignment~$2$
    by Team~$1$
  }
}

\author{Sander Cox \and Martin Kjellin \and Malin Lundberg \and Sverrir
  Thorgeirsson}

\date{\today}

\begin{document}
\maketitle

\section{Introduction}

This report describes our program to compete
on the Where's Croc game. The goal of the game is to find the crocodile Croc,
that can be located at any of 35 waterholes in Wollomunga national
park. Hopefully, he will be found before he eats the two backpackers that roam
the park. The search is performed using a \emph{hidden Markov model}
(HMM).

\section{Overall algorithm}
\label{overall}

The overall algorithm that we use to find Croc is as follows. An infinite series of
game sessions, each consisting of 100 games, is run. In the beginning of each new
game, each waterhole is assigned a probability of $\frac{1}{35}$ of being Croc's
current location; in other words we initialize with a uniform distribution. Then, the
following series of steps is repeated until Croc has been found.

The probability of Croc being currently located at each waterhole is estimated using the so-called \emph{state estimation algorithm} (described in
section~\ref{locate}). %Mention the double normalization? - I don't think we have to :) /Sverrir
When the estimation is finished, the probabilities are adjusted using
information about the locations of the backpackers. If any such location is a
positive integer (which means that the backpacker is not being eaten), we know
that Croc is currently not located at that waterhole. Consequently, the
probability for that waterhole is set to zero. If, on the other hand, the
location of a backpacker is a negative integer (which means that the backpacker is
being eaten), we know for certain that Croc is located at that waterhole. Consequently, the
probability for the corresponding waterhole is set to one, while the
probabilities for all other waterholes are set to zero.

In general, we would like to move towards the currently most probable
waterhole. However, if no waterhole has a probability that is particularly high in comparison to the others, it is hard to
know for sure which waterhole is actually the most probable. Therefore, we
decided that if the maximum probability over all the waterholes is lower than 0.15, we don't bother
moving towards the corresponding waterhole. Instead, we will move towards
the closest (before the latest pair of moves) of three waterholes that are
located at the center of the national 
park (as indicated by the graph structure). Moving in this way means that we will be fairly
close to all waterholes. This will likely be more beneficial than moving
towards the currently most probable one, since that waterhole might be located
far away from the one that will be considered the most probable one in the
next iteration. We indicate the decision to
move towards one of the central waterholes by setting it as the most probable
waterhole.

Then, a standard breadth-first search is performed in order to find a shortest path
from our current location to the currently most probable waterhole. If the
length of the path (in number of waterholes) is 1, our current location is the
most probable waterhole, so we search directly. The probability of our current
location is set to zero, since we know that if Croc is not found during this
search and another iteration must thus be performed, Croc's correct location
is not our current one. If the length of the path is two, we make one move
along the shortest path before we search and set the probability of that
waterhole to zero. If the length of the path is more than two, we just move
two steps along the shortest path in order to come closer to Croc. If Croc has
not been found (as indicated by the return value of moves and searches), the
game normally continues with a new probability estimation for each of the
waterholes. However, if the
average score of the current session is above a certain threshold that
depends on the number of played games, the session is aborted (as it will
probably not improve the record and thus is a waste of time) and a new one is started.

\section{Estimating Croc's Location}
\label{locate}

In a HMM, the state of the world at time $t$ is
described by a state variable $S_t$. We cannot observe this state
directly. Instead, we have access to a number of observation (evidence)
variables $O^x_t$ which indirectly provide us with information about the state
of the world.

There are two types of probabilities in a HMM: transition probabilities and
observation probabilities. %Check these terms
Given that the world was in state $S_{t-1}$ at time $t-1$, it
will be in state $S_t$ at time $t$ with transition probability
$P(S_t|S_{t-1})$. Given that the world is in state $S_t$, we will observe
$O^x_t$ with observation probability $P(O^x_t|S_t)$.

In this case, the state of the world is Croc's current location (waterhole
number), so $S_t \in \{1, 2, \dots, 35\}$. %check indexing
The observation variables $O^c_t$, $O^s_t$, and $O^a_t$ are the current
readings (of calcium, salinity, and alkalinity, respectively) from the sensor
that Croc is carrying.

We want to calculate $P(S_t|S_{t-1}, O^c_t, O^s_t, O^a_t)$ for each state,
that is, the probability of Croc currently lurking at that waterhole
given his previous location and the current sensor readings. This probability
is proportional to
\begin{equation*}
P(O^c_t|S_t)P(O^s_t|S_t)P(O^a_t|S_t)P^*(S_t|S_{t-1})
\end{equation*}
where
\begin{equation*}
P^*(S_t|S_{t-1}) = \sum\limits_{i=1}^{35} P(S_t|S_{t-1}=i)P(S_{t-1}=i)  
\end{equation*}
(that is, a sum over all possible previous states). % Can this be
                                % corrected/simplified? Not all states are
                                % actually possible previous states, something
                                % which we use in the program.

The recursive character of
the transition probabilities calls for a base case, that is, a value $S_0$ for
each state. As mentioned in section~\ref{overall}, we set $P(S_0) =
\frac{1}{35}$ for all waterholes. That is, we assume
that the probability is equally distributed over all waterholes.

The observation probabilities must be calculated from the sensor
readings. Since we know, for each waterhole, the expectations $\mu$ and the
standard deviations $\sigma$ of the distributions from which the readings were
drawn, we can calculate the value of the probability density function
$f(O^x_t)$ for each observation $O^x_t$ using the formula
\begin{equation*}
  f(O^x_t) = \frac{1}{\sigma \sqrt{2 \pi}}e^{-\frac{(O^x_t-\mu)^2}{2\sigma ^2}}
\end{equation*}
In a slight abuse of probability theory, these values can then be used as
observation probabilities. That is, we assume that
\begin{equation*}
  P(O^x_t|S_t) = f(O^x_t)
\end{equation*}

The values calculated as
\begin{equation*}
  P(O^c_t|S_t)P(O^s_t|S_t)P(O^a_t|S_t)P^*(S_t|S_{t-1})
\end{equation*}
are not probabilities, that is, their sum is not 1. Therefore, they are
normalized by being divided by the sum of the calculated values for all
states. The results of the normalization are the desired probabilities $P(S_t|S_{t-1}, O^c_t, O^s_t, O^a_t)$.

\section{Discussion}

Our program achieves a fairly good result. As of the time of writing, our
record score is 11.28. We think, however, that there is room for improvements:
\begin{itemize}
\item Weighing the probabilities for different waterholes using path length
  might be a good idea, at least if the difference in probabilities between
  the most current location and the others is small. That is, if Croc's currently most probable location is
  very distant from our current location, we might better decide to move
  towards some closer, but slightly less probable, location, thereby hopefully
  reducing the risk of making expensive mistakes.
\item Since the graph that we deal with is small and its structure fixed, and since we use standard breadth-first
  search, there is no real need to repeatedly search the graph for shortest
  paths. Instead, the shortest path between each pair of nodes could be
  calculated and stored once and for all. This would speed up the program,
  making it possible to play more games during a certain period of time.
  \item Sometimes, the player is located at the waterhole which we have established to be Croc's most likely location.
  As our code is written at the moment, we will in the first step search the location and in the second step stay put -
  but sometimes we are unlucky and do not find Croc and hence it would be better to do something useful
  with that second step. We have two ideas in particular, first one being to move to the neighbour of our waterhole 
  which has the highest probability of containing Croc, alternatively find the waterhole on the whole graph that 
  is most likely given our current information to contain Croc and then take the first step towards it after finding 
  the shortest path.
  b
\end{itemize}

\end{document}
